\documentclass[11pt, a4paper]{report}
\usepackage[utf8]{inputenc}

\usepackage[UKenglish]{babel}
\usepackage{csquotes}
\usepackage{enumitem}
\usepackage{fancyhdr}
\usepackage{float}
\usepackage{lastpage}
\usepackage{mathtools}
\usepackage{siunitx}
\usepackage{xcolor}

% Bibliography and references.
\usepackage[backend=biber, style=ieee]{biblatex}
\addbibresource{manual.bib}

% Set image directory.
\usepackage{graphicx}
\graphicspath{{images/}}

% Diagrams.
\usepackage{tikz}
\usetikzlibrary{shapes.geometric, arrows.meta, positioning}

% Additional keys for graphics scaling, rotating, clipping, etc. exported to \includegraphics.
\usepackage[export]{adjustbox}

% Replace paragraph indenting with newlines.
\usepackage{parskip}

% Document version history.
\usepackage[nochapter, tablegrid]{vhistory}

% Set main font first.
\usepackage[no-math]{fontspec}
\setmainfont{OpenSans}[
  Path = ./fonts/,
  Extension = .ttf,
  UprightFont = *-Regular,
  BoldFont = *-Bold,
  ItalicFont = *-Italic,
  BoldItalicFont = *-BoldItalic,
  Ligatures = TeX
]

% Load unicode-math after fontspec.
\usepackage[mathrm=sym]{unicode-math}
\setmathfont{Latin Modern Math}

% Load after babel and after fonts have been selected.
\usepackage[babel]{microtype}

% Load after most things.
\usepackage[pdfusetitle]{hyperref}

% Load after hyperref.
\usepackage[noabbrev]{cleveref}
\usepackage[margin=2cm]{geometry}

% Acronym management.
\usepackage{array}
\usepackage{acro}
\acsetup{make-links}

% Increase line spread.
\usepackage{setspace}
\setstretch{1.1}

% Increase table and other array vertical padding.
\renewcommand{\arraystretch}{1.25}

% Tikz styles for flowcharts.
\tikzset{
  startstop/.style={
    rectangle, rounded corners,
    minimum width=3cm, minimum height=1cm,
    align=center, draw=black, fill=green!30
  },
  process/.style={
    rectangle,
    minimum width=3cm, minimum height=1cm,
    align=center, draw=black, fill=blue!15
  },
  decision/.style={
    diamond, aspect=1.5,
    minimum width=3cm, minimum height=1cm,
    text width=3cm,
    align=center, draw=black, fill=orange!30
  },
  note/.style={
    rectangle, rounded corners, dashed,
    text width=5cm, font=\footnotesize,
    align=left, draw=black, fill=gray!10
  },
  arrow/.style={
    thick, ->, >=stealth
  },
  note arrow/.style={
    dashed, -, thin
  }
}

\usepackage{titlesec}
\titleformat{\chapter}{\normalfont\huge\bfseries}{\thechapter}{1em}{}
\titlespacing*{\chapter}{0pt}{50pt plus 10pt minus 5pt}{15pt plus 5pt}

\makeatletter
\fancypagestyle{plain}{%
  \fancyhf{}%
  \fancyfoot[L]{\small\color{gray} \@title\ --- York Formula Student --- v\vhCurrentVersion}%
  \fancyfoot[R]{\small\color{gray} Page \thepage\ of \pageref*{LastPage}}%
  \renewcommand{\headrulewidth}{0pt}%
  \renewcommand{\footrulewidth}{0pt}%
}
\makeatother
\pagestyle{plain}

\title{Electronics Manual}
\date{\vhCurrentDate}
\author{
  Owen Smith\\
  \texttt{ldp520@york.ac.uk}
}

\newcommand{\addschematic}[3][1.0]{%
  \begin{figure}[H]%
    \centering%
    \includegraphics[fbox, width=#1\textwidth]{figures/#2.pdf}%
    \caption{#3}%
    \label{fig:#2}%
  \end{figure}%
}

\newcommand{\todo}[1]{\textit{#1}}
\newenvironment{todoblock}{\itshape}{}
\newcommand{\iic}{I\textsuperscript{2}C}

% Possessive acronym forms.
\DeclareAcroEnding{possessive}{'s}{'s}
\NewAcroCommand\acg{m}{\acropossessive\UseAcroTemplate{first}{#1}}
\NewAcroCommand\acsg{m}{\acropossessive\UseAcroTemplate{short}{#1}}
\NewAcroCommand\aclg{m}{\acropossessive\UseAcroTemplate{long}{#1}}

\DeclareAcronym{air}{
  short = AIR,
  long = accumulator isolation relay
}
\DeclareAcronym{bms}{
  short = BMS,
  long = battery management system,
  short-plural = es
}
\DeclareAcronym{can}{
  short = CAN,
  long = controller area network
}
\DeclareAcronym{ts}{
  short = TS,
  long = tractive system
}
\DeclareAcronym{tsal}{
  short = TSAL,
  long = tractive system active light
}

\begin{document}
\pagenumbering{roman}

\makeatletter
\begin{titlepage}
  \vspace*{\fill}
  \centering
  {\fontsize{50}{60}\selectfont \bfseries \@title}\\[4ex]
  {\Huge York Formula Student}\\[4ex]
  \includegraphics[width=0.3\textwidth]{yfs-logo.png}
  \hspace{5mm}
  \includegraphics[width=0.3\textwidth]{uoy-logo.png}\\[4ex]
  {\Large v\vhCurrentVersion}\\[2ex]
  {\large
    \begin{tabular}[t]{c}%
      \@author
    \end{tabular}\par
  }
  \vspace*{\fill}
\end{titlepage}
\makeatother

\clearpage
\tableofcontents
\listoffigures
\listoftables

\chapter*{Acronyms}
\addcontentsline{toc}{chapter}{Acronyms}
\label{cha:acronyms}
\printacronyms[heading=none, template=tabular]

\chapter*{Version History}
\addcontentsline{toc}{chapter}{Version History}
\label{cha:versions}
\begin{versionhistory}
  \vhEntry{0.1}{\today}{OS}{Development version}
\end{versionhistory}

% Switch back to arabic numerals for main content.
\clearpage
\pagenumbering{arabic}

\chapter{Accumulator Design}
\begin{itemize}
\item Uses Molicel-P30B 18650 3 Ah cells
\item Subdivided into five segments
\item Each segment is made up of a cell configuration of 24s4p
\item Which results in a maximum segment voltage of 100.8 volts and segment energy of around 4.3 MJ, which is compliant with EV5.3.2
\end{itemize}

\chapter{Precharge}
The precharge is a circuit responsible for charging the large capacitance of the inverter through a precharge resistor in order to avoid a large current spike when the \acp{air} close, which could weld or otherwise damage them.
It has a secondary function of sensing and outputting the \ac{air} and precharge relay intended and actual (mechanical) states to the \acg{tsal} logic.
The key specifications of the precharge circuit are as follows:
\begin{table}[H]
  \centering
  \begin{tabular}{|l|c|}
    \hline
    \textbf{Specification} & \textbf{Value} \\
    \hline
    Low voltage input & \qty{12}{\V} \\
    \hline
    High voltage input range & \qtyrange{36}{480}{\V} \\
    \hline
    Precharge resistance & \qty{1}{\kohm} \\
    \hline
    Precharge time to 95\% @ \qty{300}{\V} \& \qty{200}{\uF} & \qty{600}{\ms} \\
    \hline
  \end{tabular}
  \caption{Precharge Specifications}
  \label{tbl:precharge-specs}
\end{table}

The main precharge operation is summarised in \cref{fig:precharge-hl}.
Voltage from the battery goes through fuse \texttt{F1} and into one contact of the precharge relay \texttt{K1}, which is a dedicated relay for precharging -- rated for high inrush and high voltage, but (relatively) low continuous current \cite{omron:G9EJH}.
When the relay closes, current flows from the battery to the tractive system (inverter) which allows its capacitance to be charged in a controlled manner via the precharge resistor \texttt{R1}, which is an anti-pulse resistor rated to withstand the initial high inrush power of precharging \cite{alpr1000}.

\addschematic[0.4]{precharge-hl}{High Level Precharge Operation}

\section{Voltage Sensing}
In order to ensure correct precharge operation, the precharge circuit must monitor the voltage before and after the precharge resistor, which are \texttt{SENSE1} and \texttt{SENSE2} in \cref{fig:precharge-hl}.
This allows detection of any errors such as bad connections and welded relays, and ensures that the precharge completes to the correct voltage in a suitable timeframe.
Since the \ac{bms} already monitors the battery voltage (net \texttt{ACC} in \cref{fig:precharge-hl}), the precharge circuit senses the input voltage just after the precharge relay, which doubles as the method of calculating the precharge relay's actual (mechanical) state (see \cref{sec:precharge-actual-state}).
Whilst voltage on each side could be retrieved from the \ac{bms} and the inverter over \ac{can} bus, the precharge circuit monitors them both independently in order to to make its operation non-reliant on the \ac{can} network (other than the start procedure, which could be changed if desired), as well as to be in complete control of the sensing period and ensure that sampling of both voltages is performed at the same instant.

\addschematic{precharge-voltage-sensing}{Precharge Voltage Sensing}

\Cref{fig:precharge-voltage-sensing} shows the voltage sensing circuit based around the \texttt{AMC1351} isolated amplifier.
The input high voltage enters a potential divider which divides the voltage by a ratio of:
\begin{equation}
  \frac{V_{in}}{V_{scaled}} = \frac{6 \cdot \qty{750}{\kohm} + \qty{47}{\kohm}}{\qty{47}{\kohm}} \approx 96.7
\end{equation}
The resistors of the potential divider are in a \texttt{1206} package to ensure they have a high voltage rating headroom.
Since the \texttt{AMC1351} has a maximum operating input voltage of \qty{5}{\V}, this sets the upper limit on the high voltage input:
\begin{equation}
  V_{max} = \qty{5}{\V} \cdot 96.7 \approx \qty{483.5}{\V}
\end{equation}
The \texttt{AMC1351} has a linear transfer function with a gain of \qty[per-mode=symbol]{0.4}{\V\per\V}, meaning the input to output voltage relationship after single-ended conversion with the \texttt{TLV6002} op-amp is described by:
\begin{equation}
  V_{out} = \left(\frac{V_{in}}{96.7} \cdot 0.4 \right) + \qty{1.65}{\V}
\end{equation}
Since the \texttt{STM32F103} has 12 bits of ADC resolution up to \qty{3.3}{\V}, there are \(\frac{4096}{2} = 2048\) ADC counts in the output range of \qtyrange{1.65}{3.3}{\V}.
Hence there is a high voltage resolution of:
\begin{equation}
  V_{res} = \frac{\qty{480}{\V}}{2048} \approx \qty{234}{\mV}
\end{equation}

Note that input side and output side power supplies of the \texttt{AMC1351} are completely isolated with separate grounds and a separate \texttt{+5VA} supply which comes from an on-board DC-DC switching converter.
This keeps the high voltage input section separate from the logic.

\section{Relay Control and States}
\label{sec:precharge-actual-state}
A comparator tapped from the divided input at \texttt{SENSE1} is used to provide the precharge relay's actual state using hardware only.
\Cref{fig:precharge-actual-state} shows this circuit using a comparator on the high voltage side which feeds an optocoupler.
The potential divider formed by \texttt{R1} and \texttt{R2} sets the threshold to:
\begin{equation}
  V = \frac{\qty{1.2}{\kohm}}{\qty{1.2}{\kohm} + \qty{18}{\kohm}} \cdot \qty{5}{\V} \cdot 96.7 \approx \qty{30.2}{\V}
\end{equation}
This defines the lower limit of the high voltage input as specified in \cref{tbl:precharge-specs}.
The circuit is designed so that a low output indicates that the relay is closed, and a high output indicates that the relay is open.
If the precharge relay is open, then \(- > +\) and the \texttt{LM393} pulls its output to ground, activating the optocoupler's LED and pulling the output high through the optotransistor.
Otherwise if the relay is closed, or if the comparator or optocoupler are unpowered or otherwise not functioning, then the comparator's output is floating, the LED and transistor remain off, and \texttt{R5} pulls the output low.

\addschematic{precharge-actual-state}{Precharge Relay Actual State Sensing}

Diode \texttt{D3} shown in \cref{fig:precharge-hl} is used to ensure that any tractive system voltage doesn't affect the precharge relay actual state detection.
At first, a Schottky diode looks attractive since a lower forward voltage causes less error when precharging.
However the diode must be able to withstand the full reverse voltage of the tractive system (up to \qty{480}{\V}).
In addition, the reverse leakage current must be low to avoiding introducing error in the accumulator side voltage measurement.
Hence a \texttt{BYG20J} diode is used, which has a worst-case reverse leakage current of \qty{10}{\uA} \cite{vishay:BYG20J}.
Using the Thévenin equivalent resistance of the high voltage input divider, calculated as:
\begin{equation}
  R_{th} = \frac{6 \cdot \qty{750}{\kohm} \cdot \qty{47}{\kohm}}{6 \cdot \qty{750}{\kohm} + \qty{47}{\kohm}} \approx \qty{46.51}{\kohm}
\end{equation}
The maximum voltage measurement error induced by the reverse leakage can then be calculated with:
\begin{equation}
  V_{error} = I_R R_{th} = \qty{10}{\uA} \cdot \qty{46.51}{\kohm} \approx \qty{465}{\mV}
\end{equation}
The maximum power dissipation of the diode can be calculated assuming maximum precharge current (from the highest allowed input voltage), and a reasonable estimate of the diode's forward voltage:
\begin{equation}
  P_{max} = \frac{\qty{480}{\V}}{\qty{1}{\kohm}} \cdot \qty{1.3}{\V} = \qty{624}{\mW}
\end{equation}
Which are both well within acceptable limits.

\todo{Talk about isolated DC-DC and why it's powered from TSAL.}

Actual state detection for the \acp{air} is much simpler since the secondary contacts can be used to infer the primary contact state.
\todo{Add AIR driving circuit.}

\section{State Machine}
The precharge software uses a finite state machine.
\Cref{fig:precharge-states} shows an overview of the various states.
The error state is currently non-leavable and the process must be reset by cycling low voltage power.

\begin{figure}[H]
  \centering
  \begin{tikzpicture}[node distance=1cm]
    \begin{scope}[local bounding box=Main]
      \node (precheck) [startstop] at (0, 0) {Precheck};
      \node (standby) [process, below=of precheck] {Standby};
      \node (precharge) [process, below=of standby] {Precharge};
      \node (precharge-hold) [process, below=of precharge] {Precharge Hold};
      \node (active) [process, below=of precharge-hold] {Active};
    \end{scope}
    \useasboundingbox (Main.north west) rectangle (Main.south east);
    \node (error) [process, left=1cm of precharge, fill=red!30] {Error};

    \draw [arrow] (precheck) -- (standby);
    \draw [arrow] (standby) -- (precharge);
    \draw [arrow] (precharge) -- (precharge-hold);
    \draw [arrow] (precharge-hold) -- (active);
    % \draw [arrow] (active.east) to[out=0, in=0] (precheck.east);
    \draw [arrow] (precheck.west) to[out=180, in=90] (error.north);
    \draw [arrow] (precharge.west) -- (error.east);
    \draw [arrow] (active.west) to[out=180, in=270] (error.south);
  \end{tikzpicture}
  \caption{Precharge State Machine}
  \label{fig:precharge-states}
\end{figure}

\Cref{tbl:precharge-relay-states} shows the intended relay states for each of the states.

\begin{table}[H]
  \centering
  \begin{tabular}{|c|c|c|c|c|}
    \hline
    \textbf{State} & \textbf{Discharge} & \textbf{AIR-} & \textbf{AIR+} & \textbf{Precharge} \\
    \hline
    Precheck/Standby/Error & Closed & Open & Open & Open \\
    \hline
    Precharge & Open & Closed & Open & Closed \\
    \hline
    Precharge Hold & Open & Closed & Closed & Closed \\
    \hline
    Active & Open & Closed & Closed & Open \\
    \hline
  \end{tabular}
  \caption{Precharge Relay States}
  \label{tbl:precharge-relay-states}
\end{table}

\subsection{Precheck}
\begin{figure}[H]
  \centering
  \begin{tikzpicture}[node distance=0.75cm]
    \begin{scope}[local bounding box=Main]
      \node (start) [startstop] {Precheck};
      \node (open-all) [process, below=of start] {Open all relays};
      \node (check-actual) [decision, below=of open-all, font=\footnotesize, minimum width=5cm] {Are all relay actual states open?};
      \node (check-acc) [decision, below=of check-actual, minimum width=5cm] {Is \(V_{acc} = \qty{0}{\V}\)?};
      \node (wait-discharge) [process, below=of check-acc] {Wait for discharge (\(V_{ts} = \qty{0}{\V}\))};
      \node (move-standby) [process, below=of wait-discharge] {Move to standby state};
    \end{scope}
    \useasboundingbox (Main.north west) rectangle (Main.south east);
    \node (open-all-note) [note, right=of open-all] {This ensures the discharge relay is closed by opening the MCU shutdown relay};
    \node (check-actual-error) [process, left=1cm of check-actual, fill=red!30] {Move to error state};
    \node (check-acc-error) [process, left=1cm of check-acc, fill=red!30] {Move to error state};
    \draw [arrow] (start) -- (open-all);
    \draw [note arrow] (open-all) -- (open-all-note);
    \draw [arrow] (open-all) -- (check-actual);
    \draw [arrow] (check-actual) -- node[anchor=south] {no} (check-actual-error);
    \draw [arrow] (check-actual) -- node[anchor=west] {yes} (check-acc);
    \draw [arrow] (check-acc) -- node[anchor=south] {no} (check-acc-error);
    \draw [arrow] (check-acc) -- node[anchor=west] {yes} (wait-discharge);
    \draw [arrow] (wait-discharge) -- (move-standby);
  \end{tikzpicture}
  \caption{Precheck Flow}
  \label{fig:precheck-flow}
\end{figure}

\subsection{Standby}
The standby state currently just waits for established communication with the \ac{bms} in order to crosscheck precharge voltage (\(V_{acc}\)) against the battery voltage.
In the future, this could be extended to wait for a start request over \ac{can}, along with a non-latching error state.

\begin{figure}[H]
  \centering
  \begin{tikzpicture}[node distance=0.75cm]
    \node (start) [startstop] {Standby};
    \node (wait-can) [process, below=of start] {Wait for \ac{bms} voltage over \ac{can}};
    \node (move) [process, below=of wait-can] {Move to precharge state};
    \draw [arrow] (start) -- (wait-can);
    \draw [arrow] (wait-can) -- (move);
  \end{tikzpicture}
  \caption{Standby Flow}
  \label{fig:standby-flow}
\end{figure}

\subsection{Precharge}
The precharge state is responsible for monitoring the tractive system voltage as the precharge relay is closed.
Note that the relay actual states are not checked here since that is deemed the job of the \ac{tsal} for the most part.
Any relay failure should be detected by the voltage measurement.

\begin{figure}[H]
  \centering
  \begin{tikzpicture}[node distance=0.75cm]
    \begin{scope}[local bounding box=Main]
      \node (start) [startstop] {Precharge};
      \node (close-mcu) [process, below=of start] {Close MCU shutdown relay};
      \node (check-shutdown) [decision, below=of close-mcu, font=\footnotesize] {Is shutdown ready?};
      \node (close-precharge) [process, below=of check-shutdown] {Close AIR- and precharge};
      \node (wait-precharge) [process, below=of close-precharge] {Wait \qty{30}{\ms} for relay close};
      \node (check-acc) [decision, below=of wait-precharge, font=\small] {Is \(V_{acc}\) in range?};
      \node (calculate) [process, below=of check-acc] {Calculate the expected \(V_{ts}\) for\\ the current instant};
      \node (check-ts) [decision, below=of calculate, font=\small] {Is \(V_{ts}\) in range?};
      \node (check-done) [decision, below=of check-ts, font=\small] {Is precharge done?};
      \node (move-hold) [process, below=of check-done] {Move to precharge hold state};
    \end{scope}
    \useasboundingbox (Main.north west) rectangle (Main.south east);
    \node (check-shutdown-error) [process, left=1cm of check-shutdown, fill=red!30] {Move to error state};
    \node (check-acc-error) [process, left=1cm of check-acc, fill=red!30] {Move to error state};
    \node (check-ts-error) [process, left=1cm of check-ts, fill=red!30] {Move to error state};
    \draw [arrow] (start) -- (close-mcu);
    \draw [arrow] (close-mcu) -- (check-shutdown);
    \draw [arrow] (check-shutdown) -- node[anchor=south] {no} (check-shutdown-error);
    \draw [arrow] (check-shutdown) -- node[anchor=west] {yes} (close-precharge);
    \draw [arrow] (close-precharge) -- (wait-precharge);
    \draw [arrow] (wait-precharge) -- (check-acc);
    \draw [arrow] (check-acc) -- node[anchor=south] {no} (check-acc-error);
    \draw [arrow] (check-acc) -- node[anchor=west] {yes} (calculate);
    \draw [arrow] (calculate) -- (check-ts);
    \draw [arrow] (check-ts) -- node[anchor=south] {no} (check-ts-error);
    \draw [arrow] (check-ts) -- node[anchor=west] {yes} (check-done);
    \draw [arrow] (check-done.east) to[out=0, in=0] node[anchor=west] {no} (check-acc.east);
    \draw [arrow] (check-done) -- node[anchor=west] {yes} (move-hold);
  \end{tikzpicture}
  \caption{Precharge Flow}
  \label{fig:precharge-flow}
\end{figure}

\subsection{Precharge Hold}
The precharge hold state ensures a smooth transition between the precharge relay opening and the positive AIR closing, by adding a period of overlap between them.

\begin{figure}[H]
  \centering
  \begin{tikzpicture}[node distance=0.75cm]
    \node (start) [startstop] {Precharge Hold};
    \node (close) [process, below=of start] {Close AIR+};
    \node (delay) [process, below=of close] {Wait for \qty{500}{\ms}};
    \node (open) [process, below=of delay] {Open precharge};
    \node (move-active) [process, below=of open] {Move to active state};
    \draw [arrow] (start) -- (close);
    \draw [arrow] (close) -- (delay);
    \draw [arrow] (delay) -- (open);
    \draw [arrow] (open) -- (move-active);
  \end{tikzpicture}
  \caption{Precharge Hold Flow}
  \label{fig:precharge-hold-flow}
\end{figure}

\subsection{Active}
\begin{figure}[H]
  \centering
  \begin{tikzpicture}[node distance=0.75cm]
    \begin{scope}[local bounding box=Main]
      \node (start) [startstop] {Active};
      \node (check) [decision, below=of start, font=\footnotesize] {Is shutdown still ok?};
      \node (move-error) [process, below=of check, fill=red!30] {Move to error state};
    \end{scope}
    \useasboundingbox (Main.north west) rectangle (Main.south east);
    \draw [arrow] (start) -- (check);
    \draw [arrow] (check) -- node[anchor=west] {no} (move-error);
    \draw [arrow] (check.west) to[out=180, in=180] node[anchor=east] {yes} (start.west);
  \end{tikzpicture}
  \caption{Active Flow}
  \label{fig:active-flow}
\end{figure}

\section{Precharge Operation}

\subsection{Normal Operation}

\section{Failure Modes}

\section{CAN Communication}

\chapter{Battery Management System}
The Battery Management System (BMS) is an integral part of the overall accumulator design of the YFS car, which implements a distributed design.
The primary job of the BMS is to monitor cell voltages and temperatures and provide cell balancing capability during charging.

\section{Applicable Rules}
As per the FSUK 2025 rules, the BMS must:
\begin{itemize}
\item\textbf{EV5.8.1:} Be monitoring whenever the low-voltage system or charging is active.
\item\textbf{EV5.8.3:} Continuously measure:
  \begin{itemize}
  \item All cell voltages;
  \item The temperature of at least 30\% of all cells;
  \item The total tractive system current.
  \end{itemize}
\item\textbf{EV5.8.7:} Switch off the tractive system via the shutdown circuit if critical voltage, temperature, or current values are reached and are persistent for more than:
  \begin{itemize}
  \item 500 ms for voltage and current values;
  \item 1 s for temperature values.
  \end{itemize}
\item\textbf{EV5.8.10:} Fail-safe.
  In particular, all signals are system critical signals, meaning the loss of any measurement signal must result in an opened shutdown circuit.
\item\textbf{EV5.8.11:} Have individually disconnectable current sensors, temperature sensors, and cell voltage taps.
\item\textbf{EV5.8.12:} Be able to display all measured values at any time.
\end{itemize}

In general, the BMS does not need to concern itself with rule \textbf{EV1.2.2}:
\begin{displayquote}
  High Current Path -- any path of a TS circuitry that, during normal operation, carries more than 1A.
\end{displayquote}
Since no significant current, other than cell balance current which is under one amp, flows through the BMS.
In particular, this means that the BMS connections are not required to be positively locking.

\section{High-Level Design}
Although the accumulator is physically divided into five segments, from the perspective of the BMS, there are ten segments.
This design choice was made for two primary reasons:
\begin{enumerate}[label=(\arabic*)]
\item Each distributed BMS board is limited to handling a maximum of 50 volts, making the system both safer to work with and more practical for testing and development.
\item By dividing the accumulator into ten segments, each BMS board is responsible for monitoring just 12 series cells, which is more practical compared to handling 24 cells per board.
\end{enumerate}

Each of these ten segments is managed by a distributed segment board, all of which are connected to a master BMS via an isolated \iic{} bus.
The segment boards operate in a passive manner, responding to requests from the master BMS to perform a measurement of all cell voltages and temperatures.
When the low-voltage system is active, the master BMS polls each segment board at a regular interval.
In the event of a segment BMS failure, \iic{} communication would be lost, but the system fails safe with the master BMS opening the shutdown circuit.

Charging control is managed by the master BMS through a GPIO output.
During charging, the segment boards enter a continuous measurement mode, which acts mostly autonomously.
In this mode, the cell balancing algorithm is enabled.
A periodic heartbeat message is sent from the master BMS to each distributed board to confirm that charging is still active, and a response from each segment BMS to the master BMS confirms that the cell voltages and temperatures are still within safe limits.

Current measurement is also performed by the master BMS, through the use of Hall-effect current sensors which measure the total tractive system current.
A connection to the car's CAN bus allows measurement logging and configuration of the BMS.
This CAN connection is non-critical, since the shutdown mechanism is implemented through a dedicated digital output.

\section{Segment-Level Design}
This section covers the distributed segment design of the BMS.

\begin{figure}[H]
  \centering
  \includegraphics[width=\linewidth]{bms.png}
  \caption{Segment BMS PCB Render}
  \label{fig:bms-pcb}
\end{figure}

Cell voltage taps are connected via a pluggable terminal block.
Cell voltage measurements are provided by three main components:
\begin{itemize}
\item The MAX14920 12-cell measurement analog frontend
\item The MAX11163 16-bit SAR ADC
\item The REF6145 high-precision 4.5 volt reference
\end{itemize}
This selection of components allows for sub-millivolt cell voltage measurement accuracy.

The microcontroller used is the STM32F103.
Its main job is to control both the analog frontend and ADC via SPI, and communicate to the master BMS over \iic{}.
A buck converter steps-down the whole segment voltage (44 volts nominal; 50 volts maximum) to 3.3 volts to power the microcontroller and logic components of the analog frontend and ADC.

\iic{} isolation is provided by the ADuM1252 ultra-low power bidirectional \iic{} isolator IC.
This IC provides strong galvanic isolation, and allows for hot-plugging a BMS onto an active bus.

Each 4p cell set in the 12s series string has a balancer.
This is a small circuit controllable by software which bleeds off cell energy as heat during charging if cell voltages start to become unbalanced.

\begin{figure}[H]
  \centering
  \includegraphics[width=0.6\linewidth]{balance.png}
  \caption{Cell Balance Circuit}
  \label{fig:bms-balance-circuit}
\end{figure}

The balance circuit consists of a high-power 10 ohm resistor which is shunted across cell N and cell N-1 by the MOSFET when the digital balance signal is high.
This balance signal is provided by the MAX14920 analog frontend IC, which has an internal pull-down so no external gate-source resistor is necessary.
An LED is also present to visually indicate when balancing of each cell is active.
A value of 10 ohms was chosen to achieve a balance current of about 400 mA, which results in around 100 mA per cell.

The BMS provides cell temperature monitoring via simple NTC thermistors.
Up to 20 external thermistors can be connected, which is more than the 15 minimum required by rule EV5.8.3 for our 12s4p configuration.
Additionally, there are three onboard thermistors spaced evenly around the balance section in order to monitor the temperature of the bleed resistors.

\begin{figure}[H]
  \centering
  \includegraphics[width=\linewidth]{thermistor.png}
  \caption{Equivalent Thermistor Circuit}
  \label{fig:bms-thermistor-circuit}
\end{figure}

The above figure shows the equivalent schematic of the thermistor connections.
The thermistors are arranged in groups of three, with each group being switchable by the Q1 MOSFET.
T1-T3 are the auxillary single-ended analog inputs of the MAX14920 analog frontend IC, protected by the 1k series resistors.
R4-R6 are pull-up resistors forming a potential divider with the thermistors.
Q1 has a gate resistor and is driven by the main STM32F103 microcontroller.

Importantly, the precision reference used for cell voltage measurement is not used for temperature measurement, but rather the regular 3V3 power rail.
This was done to reduce complexity since the accuracy of the thermistor measurements is not super critical, and the nonlinearities and tolerances from the thermistors themselves will result in more error than the ADC reference voltage.

As both the cell tap and thermistor connections are system critical, the system must fail safe in the event of either an open or a short circuit.
The following table describes the possible scenarios.

\begin{table}[H]
  \centering
  \begin{tabular}{|p{0.3\linewidth}|p{0.7\linewidth}|}
    \hline
    \textbf{Failure} & \textbf{Mitigation} \\
    \hline
    Cell tap disconnected & TBD \\
    \hline
    Cell tap shorted to another cell/ground & The cell voltage would become out of range for measurement, which would be detected.
                                              In addition, a large current may flow causing the individual cell fuse to blow, which would effectively look like a disconnected cell tap to the BMS (see above). \\
    \hline
    Thermistor disconnected or shorted to supply voltage & The signal would read 3.3 volts, which is above the plausible range. \\
    \hline
    Thermistor shorted to ground & The signal would read 0 volts, which is below the plausible range. \\
    \hline
  \end{tabular}
  \caption{Segment BMS SCS Compliance}
  \label{tbl:segment-bms-scs-compliance}
\end{table}

\section{Current Unknowns}
\begin{itemize}
\item How to power the main BMS board.
\item How EV3.2.5 fits in with the current measurement of the master BMS.
\end{itemize}

\printbibliography
\end{document}
